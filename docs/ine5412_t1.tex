\documentclass[12pt]{article}

\usepackage[brazil]{babel}
\usepackage[T1]{fontenc}
\usepackage[a4paper, margin=1.5cm]{geometry}
\usepackage[colorlinks, urlcolor=blue, citecolor=red]{hyperref}
\usepackage[utf8]{inputenc}
\usepackage{listings}

\lstdefinelanguage{cc}{
  language = C,
  basicstyle=\ttfamily\footnotesize,
  keywordstyle=\color{blue},
  morekeywords = {uint8_t, uint16_t, intptr_t, __builtin_popcount, SIZE},
}

\title{\textbf{Verificador de Sudoku \emph{Multithread}}}
\author{
  Gustavo F. Guimarães, Gustavo Zambonin, Marcello Klingelfus\thanks{
    \texttt{\{gustavo.g,gustavo.zambonin,marcello.klingelfus\}@grad.ufsc.br}
    \hfill \texttt{\href{https://github.com/zambonin/ufsc-ine5412}{src/}}
  } \\
  \small{Sistemas Operacionais I (UFSC -- INE5412)}
}
\date{}

\setlength{\parindent}{0pt}

\begin{document}

\maketitle

\section{Verificação da repetição para um conjunto de números}

\emph{Funções aludidas}: \texttt{check\_row}, \texttt{check\_col},
\texttt{check\_sqr}, \texttt{print\_errors} \\

Para evitar a maior complexidade de tempo e espaço obtida a partir de uma
solução comum para este problema, uma abordagem mais minimalista foi utilizada,
que não envolve desvios condicionais, a criação de vetores auxiliares ou a
ordenação do conjunto. É possível perceber que apenas oito bits poderiam ser
utilizados para carregar o resultado da verificação da lista de números, pois
no caso do \emph{sudoku} sempre existirá um número em sua posição correta.
Assim, apenas oito checagens seriam necessárias; entretanto, a prototipação
desta estratégia mostrou que a abordagem a seguir teria um melhor balanço entre
legibilidade do código e desempenho alcançado. \\

Sendo assim, o método desenvolvido consiste na alocação de um inteiro
\emph{unsigned} de 16 bits, onde apenas nove destes são utilizados. A cada
iteração na região selecionada, um deslocamento à esquerda é feito $n$ vezes,
onde $n$ é o número presente no vetor na posição referente à iteração. Assim,
caso uma região não apresente erros, o inteiro alocado terá valor
$(1 << 9) - 1 = 2^9 - 1 = 511$. Em Python, este método pode ser escrito como
\verb!reduce(lambda x, y: x | ((1 << y) - 1), range(1, 10))!.
Note que é possível substituir o iterável \texttt{range} por qualquer lista de
números.

\begin{figure}[htbp]
  \lstinputlisting[language=cc, linerange={91-98}]{../sudoku_verifier.c}
  \caption{Checagem de repetições em uma região quadrada do tabuleiro. Cada
    uma destas pode ser obtida a partir de um índice $i \in \{0, 8\}$,
    organizadas da esquerda para a direita, de cima para baixo.}
\end{figure}

Caso existam repetições, alguns bits não serão ativados, e deste modo, estes
podem ser contados para descobrir a quantidade de números repetidos naquela
linha. Isto é feito com uma extensão do compilador GCC (também suportada por
Clang), chamada de \texttt{\_\_builtin\_popcount} \cite{gcc}, que conta a
quantidade de bits \texttt{1} em um inteiro; desse modo, o contador de erros é
negado com uma máscara correspondente para que a contagem correta seja feita.

\begin{figure}[htbp]
  \lstinputlisting[language=cc, linerange={104-104}]{../sudoku_verifier.c}
  \caption{Implementação da contagem de erros descrita acima.}
\end{figure}

\section{Distribuição das tarefas entre \emph{threads}}

\emph{Funções aludidas}: \texttt{main}, \texttt{choose\_task},
\texttt{print\_errors} \\

Solucionar o problema proposto envolve tornar o programa verificador
concorrente. Deste modo, é necessário identificar possibilidades de divisão de
tarefas a fim de aumentar seu desempenho. Percebe-se que a única tarefa que
pode ser dividida é a de verificação de regiões diferentes, visto que existem
vinte e sete destas (nove linhas, nove colunas e nove quadrados internos). \\

O número de \emph{threads} desejadas para a execução do programa é obtido como
um argumento fornecido para o programa. É necessário apontar que o número
máximo para tal, limitado em $2^{16} - 1$ neste programa, é fundamentado pela
seguinte razão:

\begin{verbatim}
$ man pthread_create | grep "limit" -m4 -A2
       EAGAIN A system-imposed limit on the number of threads was encountered.
              There are a number of limits that may trigger this error: the
              RLIMIT_NPROC soft resource limit (set via setr‐limit(2)), which
              limits the number of processes and threads for a real user ID,
              was reached; the kernel's system-wide limit on the number of
              processes and threads, /proc/sys/kernel/threads-max, was reached
              (see proc(5)); or the maximum number of PIDs,
              /proc/sys/kernel/pid_max, was reached (see proc(5)).
$ cat /proc/sys/kernel/threads-max
61085
\end{verbatim}

O valor acima foi observado em outros computadores sem maiores variações. Com
o número escolhido, um vetor é criado para armazenar as \emph{threads}. Ao
inicializá-las, é necessário inserir seus índices como argumento para o método
que realiza a distribuição das tarefas. \\

A distribuição é implementada da seguinte maneira: um vetor de ponteiros de
função (declarado com uma extensão da linguagem C que mostra advertências em
compilações com configurações mais rigorosas) e um inteiro de 32 bits são
criados a fim de representar as tarefas disponíveis. Ao iterar sobre este vetor
e escolher uma tarefa, a \emph{thread} mudará o bit referente ao índice desta,
assim marcando-a como completa, e executará o ponteiro de função com os
argumentos necessários.  Este procedimento é sensível à condições de corrida,
pois duas \emph{threads} podem escolher a mesma tarefa e mudar o bit duas
vezes, de modo que uma terceira poderá escolhê-la novamente. Portanto,
mecanismos de exclusão mútua abrangem as partes críticas da função.

\begin{figure}[htbp]
  \lstinputlisting[language=cc, linerange={114-123}]{../sudoku_verifier.c}
  \caption{Escolha de tarefas a partir de uma `'fila`'. É possível observar
    que é necessário liberar a \emph{thread} em dois momentos diferentes
    pois, do contrário, ela nunca retornará ao fluxo de execução usual,
    impedindo a continuidade do programa. O ponteiro de função disponível em
    `tasks` é chamado com dois argumentos: a região escolhida, decidida pelo
    índice da iteração, que é sempre divisível pelo tamanho do tabuleiro; e
    o índice da \emph{thread}, convertido de um \texttt{void*} proveniente
    da criação desta.}
\end{figure}

Outra abordagem possível é uma distribuição prévia das tarefas entre as
\emph{threads}; assim, estas já saberiam de suas responsabilidades no momento
que são criadas.  Essa abordagem alternativa pode prevenir um problema onde uma
ou mais \emph{threads} estão sempre tentando obter tarefas mas falhando, assim
presas na fila de espera e não oferecendo desempenho máximo. \\

Após a contagem de erros, caso um exista um elemento repetido, é necessário
somar estes ao número total. Como duas \emph{threads} podem modificar este
contador em períodos muito próximos, verifica-se que esta região também é
crítica; portanto, também é cercada por mecanismos que impedem a contagem
errada de erros.

\bibliography{ine5412_t1}
\bibliographystyle{plain}

\end{document}
